\input{src/header}											% bindet Header ein (WICHTIG)
\usepackage{graphicx}
\usepackage{listings} % inline code snippets

\newcommand{\dozent}{Prof.  Dr.  Agnès Voisard, Nicolas Lehmann}					% <-- Names des Dozenten eintragen
\newcommand{\tutor}{Hoffman Christian}						% <-- Name eurer Tutoriun eintragen
\newcommand{\tutoriumNo}{Tutorium 3}				% <-- Nummer im KVV nachschauen
\newcommand{\projectNo}{2.Iteration}									% <-- Nummer des Übungszettels
\newcommand{\veranstaltung}{Datenbank Systeme}	% <-- Name der Lehrveranstaltung eintragen
\newcommand{\semester}{SoeSe 2017}						% <-- z.B. SoSo 17, WiSe 17/18
\newcommand{\studenten}{Ingrid Tchilibou, Emil Milanov, Boyan Hristov}			% <-- Hier eure Namen eintragen


% /////////////////////// BEGIN DOKUMENT /////////////////////////


\begin{document}
\input{src/titlepage}										% erstellt die Titelseite

\section*{Allgemein}
Link zum Projekt: \url{https://github.com/gancia-kiss/dbs_projekt}

\section*{1.Aufgabe: Datenbankschema erstellen}


\section*{2.Aufgabe:Datenbereinigung}


\section*{3.Aufgabe: Datenimport}


\section*{4.Aufgabe: Webserver}

Wir haben uns für ein Apache Webserver entschieden. Da zwei von uns eine Arch-basierte Linux Distribution haben, haben wir so den Webserver konfiguriert.

Die config-Datei bedindet sich am Archlinux-Distributionen hier: 
\lstinline|/etc/httpd/conf/httpd.conf|

Dort haben wir die \lstinline|Listen 80| zu \lstinline|Listen 127.0.0.1:5234| verändert, damit der Webserver Verbindungen nur am Port 5234 nur vom lokalen Rechner akzeptiert.

Schließlich haben wir eine test \lstinline|index.html| Datei in \lstinline|/srv/http| hinzugefügt um den Webserver zu testen.




\end{document}
