\input{src/header}											% bindet Header ein (WICHTIG)

\newcommand{\dozent}{Prof.  Dr.  Agnès Voisard, Nicolas Lehmann}					% <-- Names des Dozenten eintragen
\newcommand{\tutor}{Hoffman Christian}						% <-- Name eurer Tutoriun eintragen
\newcommand{\tutoriumNo}{Tutorium 3}				% <-- Nummer im KVV nachschauen
\newcommand{\projectNo}{1}									% <-- Nummer des Übungszettels
\newcommand{\veranstaltung}{Datenbank Systeme}	% <-- Name der Lehrveranstaltung eintragen
\newcommand{\semester}{SoeSe 2017}						% <-- z.B. SoSo 17, WiSe 17/18
\newcommand{\studenten}{IngridTchilibou,Emil,Boyan Hristov}			% <-- Hier eure Namen eintragen
% /////////////////////// BEGIN DOKUMENT /////////////////////////


\begin{document}
\input{src/titlepage}										% erstellt die Titelseite


% /////////////////////// Aufgabe 1 /////////////////////////
\section{Aufgabe: Projektdokumentation}
{\itshape Aufgabenstellung des Dozenten}



% ///////////////////// Aufgabe 2 ///////////////////
\section{Aufgabe:Explorative Datenanalyse}
\begin{enumerate}[1)]
\item {\itshape Laden von der Datein herum}
\item{angeschaut}
\item{Beschreibung von der Datensatz}
Wir haben einen Tabelle mit 10 Attributen von verschiedenen Tweets aus der America Election von 2016. Jede Tupel beschreibt genau von wem war den Tweet gepostet? was war die Nachricht von diesem Tweet war diesem Tweets weitergeleitet? war die Authenticität von dem Tweet gut(Das heißt ob der Tweet von diese Person ,der gepostet hat geschrieben war?

\end{enumerate}
	

% /////////////////////// END DOKUMENT /////////////////////////
\end{document}